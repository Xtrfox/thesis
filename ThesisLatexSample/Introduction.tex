\chapter{Introduction}

In the era of online social networking, it can be said that people are more expressive than ever since there is now an avenue to express publicly emotions (sentiments and reactions) and actions (behavior, location) in realtime \cite{deller2011twittering}. Because of this, Twitter has become a prime candidate to learn more about the daily conditions of people. The immediacy of the contents from Twitter makes it an ideal source of data for analysis in areas such as disaster controls, marketing strategization, population profiling, health emergency tracking, and more \cite{zeng2010social, ribarsky2014social}. In the context of the Philippines, a country considered to be the social networking capital of the world, the potential of Twitter posts can be actualized  as a layer to improve on existing epidemiological systems employed in the country.


% In a world where social media accounts are rapidly becoming people's virtual identity online, gathering information about a whole population can literally be as easy as a click on a mouse. Twitter is one of the top social networking websites that users visit to keep their ``friends'' informed with things happening in their life . . This research capitulates on Twitter's features as a social networking website to track posts directly or indirectly concerning the users' state of health. Personal accounts - not that of an organization such as news accounts - will be the prime focus, since this will provide the research a more population-based surveillance needed for spatio-temporal modeling. 
\textit{"Infodemiology"} is a term coined by Eysenbach when dealing with surveillance of public health-related text in the Internet \cite{Eysenbach:2009aa}. It aims to measure and track the behaviour of information online with regards to public health. Social networking websites, such as Twitter, had proved themselves to be key to creating infodemiological studies after previous successful studies \cite{nascimento2014real, Chew:2010aa, Bragazzi:2013aa,eysenbach2006infodemiology}. The primary focus of this research is the infodemiology of three (3) of the top communicable diseases - Measles, Influenza, Typhoid Fever -  in the Philippines, using Twitter \cite{dohrecord2015}. 

%In recent years, advancements have been made to track certain diseases using different health analytics tools \cite{Simpao:2014aa, Wills:2014aa}. The methodologies present in extracting data from Hospital Information Systems (HIS)

%This is not the priority for this research as these methodologies directly gathers data from hospital information systems (HIS) that are tied to hospitals' daily activities. The methodology used in the extraction of data in an HIS  will only then serve as a guide to retrieve data from social networking websites to gather relevant information for the research, and the data gathered may then be cross-referenced to actual statistical records to produce certain correlations. 

This research is aligned with the \textit{Philippine eHealth Strategic Framework and Plan 2020} that may help in strengthening existing eHealth solutions through adding to the \textit{Information Sources} pillar of the framework. Through the use of ICT, the state of eHealth in the Philippines should be improved to provide even faster universal healthcare to people. The research may pave the way for complementing the usual way of tracking diseases that usually requires work that take up a lot of time \cite{gomide2011dengue}. 

% Some methods include the use of turnaround surveys that are given to patients, and for some other diseases, such as tuberculosis, health stations that are part of a bigger health information system are set-up to aggregate data, which is the case in tracking tuberculosis in the Philippines \cite{vianzon2013tuberculosis}. 

The power of social media lies on the immediacy it has, and the amount of context a researcher could get from the strings of texts that users post. Twitter, for example, has a maximum limit of 140 characters per tweets. This limit helps in making it easier to filter subjects, sentiments, and data through the use of computational algorithms \cite{schmidt2012trending}. The extracted data may help researchers in finding out patterns in them, which may lead into creating different kinds of models based on them.

%Currently, there is no comprehensive profile done by the Philippines' Department of Health (DOH) for the Influenza cases in the country. On the other hand, the DOH publishes weekly - sometimes tri-weekly - reports on recorded Influenza-like cases through their website \cite{dohflusurvey}. In this reports, basic statistics are shown, which includes gender ratio, age groupings, case and mortality counts, and municipality segregation. These data are vital for this research as these are the ones that are cross-referenced with users' posts. 

The Department of Health (DOH) is the government entity in charge of collecting and storing statistics regarding diseases in the country. The Epidemiology Bureau (EB) is an office under the DOH whose mandate is to develop and evaluate disease surveillance systems. Currently, the data they gather are in public domain, and can be accessed easily through the DOH's website. Other sources of anonymized aggregated data can come from Electronic Medical Record (EMR) Systems being used by different private and public hospitals, if given access. The data collected, along with the spatio-temporal details (geo-location and timestamp) embedded in it, can serve as a baseline for possible correlation with the collected Twitter posts to make an infodemiological model. Using these data, the goal of this research is to produce a model that shows and predicts the conditions of various locations regarding the chosen diseases.

% As the main government entity in charge of public health, the DOH, through the epidemiology bureau, 
%This research focuses on the clusters of people in the Philippines, having a municipality as the smallest possible collective unit. 


The research is divided into two phases, which are 1) creation of a twitter collection system for health related topics and visualization on a map, and 2) the modeling and prediction of possible areas of outbreaks based on Twitter posts. The major part of the initial phase is in the collection, pre-processing, quantification, and visualization of related tweets. This phase's primary concern is creating a listener for health related words relevant to infodemiology, such as symptoms and associated outbreaks. Users' sentiments will  disregarded, as the model to be made focuses on classifying whether a tweet is infodemiological in nature or not. Bilingual keywords or phrases will be collected, which is vital since the Philippines uses two primary languages, the Filipino language and the English language. The second phase is where the modeling of the collected tweets will happen. An infodemiological model will be developed to identify possible locations of outbreaks based on current tweets through IBM's Spatio-Temporal Epidemiological Modeler.

Social networking websites had already been used by researchers to track different diseases like Human immunodeficiency virus (HIV) and Dengue \cite{Young:2014aa, gomide2011dengue}. This research would follow some methodologies used by previous studies, but this would cater more to the Philippines since the data comes from the Philippines' domain. Added to this, the factor of bilinguality may add more needed appropriations to the study since Filipinos are highly likely to use social networking websites in both English and in Filipino.

\section{Research Questions}

This research aims to answer the question ``Can social media posts from Twitter be one of the determinants of disease outbreaks within the Philippines?''

Other subquestions are as follows:
\begin{itemize}
\item What are related keywords and phrases frequently mentioned by Twitter users regarding the three diseases, especially when they are prevalent across an area?
\item What attributes of social media posts are accessible and significant to be able to produce a population surveillance report?
\item How do we interface tweets with syndromic surveillance data from existing systems of the Department of Health as well as from electronic medical records (EMR)?
\item How do we develop a validated spatio-temporal epidemiological model that shows and forecasts possible outbreak areas based on Twitter posts?
\end{itemize}
\section{Significance of the Study}

The research is highly anchored on concepts of data mining, data analytics, and data visualization, particularly focusing on social media. Social media, being a tool used by millions of people around the world, can be a good indicator of people's behavior when proper analytics is implemented. For this research, a technique for data mining and analysis of health related information will be developed to improve on the sector of public health in the Philippines. This study builds on previous work that uses social media surveillance by adding a spatio-temporal layer of abstraction. 

Added to data mining and data analytics, modeling and visualization will also be used in the development of the predictive functionality. The gathered and analyzed data from Twitter will be used as the parameters for modeling using IBM's Spatio-Temporal Epidemiological Modeler (STEM). The model that will be developed will track how diseases actually spread using an accurate simulation not only based on spatial, but also on temporal data. Added to this, the research may also pave the way for other related diseases to be predicted.

The significance of the study also comes in the social impact that it can make. The tool developed can primary help the government, particularly the Department of Health, in making decisions regarding these diseases. One concrete way to use the results of the tool is through targeted prevention. The DOH usually deploys medical missions around the Philippines, and through the use of the tool, they can easily target those areas showing high probability of the diseases' outbreak. Early detection plays a crucial role if it is used by the government. Added to this, if proven accurate, the work of producing an up-to-date disease surveillance report can be done in a more regular basis. Through doing this, information-dissemination to the public is easier. People can then be informed immediately if there are outbreaks within their community, and take necessary precautions for themselves.

All in all, the study aims to produce and validate a visualized predictive infodemiological model that may help on the reduction of the diseases' incidences within the Philippines.