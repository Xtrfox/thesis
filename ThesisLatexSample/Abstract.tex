\begin{thesisabstract}

Infodemiology is a relatively new field of study where the margin for innovation is still large. Using infodemiology, the aim of the paper is to determine whether \textbf{tweets} can be an indicator if there is a disease outbreak within a certain area in the Philippines. Different methodologies will be used to get to the result. This includes data extraction, analytics, modeling, and visualization. The modeling of the diseases will be primarily made using IBM's Spatio-Temporal Epidemiological Modeler (STEM). Once a model and a scenario is created in STEM, a simulation of possible disease outbreak locations will be shown on a map based on user-specified timeframe. The results of this research will be especially helpful to the Philippines' Department of Health, as this study complements already existing surveillance system.

\end{thesisabstract}