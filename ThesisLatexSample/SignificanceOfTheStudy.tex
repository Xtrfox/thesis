\section{Significance of the Study}

The research is highly anchored on concepts of data mining, data analytics, and data visualization, particularly focusing on social media. Social media, being a tool used by millions of people around the world, can be a good indicator of people's behavior when proper analytics is implemented. For this research, a technique for data mining and analysis of health related information will be developed to improve on the sector of public health in the Philippines. This study builds on previous work that uses social media surveillance by adding a spatio-temporal layer of abstraction. 

Added to data mining and data analytics, modeling and visualization will also be used in the development of the predictive functionality. The gathered and analyzed data from Twitter will be used as the parameters for modeling using IBM's Spatio-Temporal Epidemiological Modeler (STEM). The model that will be developed will track how diseases actually spread using an accurate simulation not only based on spatial, but also on temporal data. Added to this, the research may also pave the way for other related diseases to be predicted.

The significance of the study also comes in the social impact that it can make. The tool developed can primary help the government, particularly the Department of Health, in making decisions regarding these diseases. One concrete way to use the results of the tool is through targeted prevention. The DOH usually deploys medical missions around the Philippines, and through the use of the tool, they can easily target those areas showing high probability of the diseases' outbreak. Early detection plays a crucial role if it is used by the government. Added to this, if proven accurate, the work of producing an up-to-date disease surveillance report can be done in a more regular basis. Through doing this, information-dissemination to the public is easier. People can then be informed immediately if there are outbreaks within their community, and take necessary precautions for themselves.

All in all, the study aims to produce and validate a visualized predictive infodemiological model that may help on the reduction of the diseases' incidences within the Philippines.