\section{General Success of Medical Informatics}

Health information technology is a relatively new field that has been gaining traction over the past few decades, and an aspect of it that has received much attention is EMR (electronic medical records), wherein patient data are stored digitally rather than written down on paper (reference). While the benefits are clear -- healthcare workflow becomes efficient and resources are maximized -- the problem lies in how to connect with users. Functional systems are developed, but users do not adapt well to these -- in fact, almost 45\% of these systems are abandoned. What is more alarming is that 85\% become ultimately ineffective, and only 15\% are successful. Instead of focusing only on tne technology, the context -- the people and organizations involved -- must also be understood as thoroughly as possible \cite{premji2012implementing}.

\subsection{Medical Informatics in the Philippines}

Health information technology is already being used extensively in Asian countries such as Singapore. Computers play an essential role in the hospitals of nearly all Asian countries, but what kind of systems are involved and how well they are used varies from country to country, with a certain dependence on the government and how much budget it is willing to allocate \cite{nguyen2008analysis}. Implementation in Europe and USA are still fraught with issues as well, as there is no one-size-fits-all system, but it is even trickier in developing countries, where health crises are prevalent (e.g. AIDS and malaria), and infrastructure and staff are lacking \cite{fraser2005implementing}.

In the Philippines, medical informatics was already present since the 1980s. The most basic form of this was residents inputting patient information into IBM-compatible machines. Development proceeded more quickly in the 1990s, when the University of the Philippines Manila opened a Medical Informatics Unit composed of academics. They engaged in extensive research, but no concrete, long-standing projects were built that would involve the public. However, they did shape local medical informatics to be distinct from what is practiced in other countries by making it community-focused. In contrast to first-world countries, which have sophisticated facilities, the UP College of Medicine consistently put the community at the forefront \cite{marcelo2007health}. e-Health Philippines was also started in 1998 in order to make it more appealing for researchers to create specialty databases \cite{nguyen2008analysis}.

\subsection{Challenges in Philippines' Medical Informatics}

As of now, like other Southeast Asian countries, there have been several attempts to make medical informatics an integral part of Philippine healthcare, but these have not been as successful as hoped for because of several factors \cite{nguyen2008analysis}:.

\begin{enumerate}
    \item Lack of interest in people

    A major stumbling block is sparking the sustained interest of healthcare professionals and decision-makers such as government officials. Some of them do not realize the immense benefits that medical informatics could provide. Those who wanted to try it seemed to be motivated only by the desire for novelty and stopped using it after a while \cite{marcelo2007health},  and the number of doctors is decreasing because many of them are switching to nursing careers so they can migrate abroad \cite{nguyen2008analysis}.

    Healthcare providers can also be reluctant to embrace new technology, especially those belonging to the older generation, as they are already comfortable with the paper-based system. They may also have never used computers before and may be anxious or fearful ("I have a computer at home but I don't really turn it on because I'm scared it might explode.") \cite{premji2012implementing}. However, they acknowledge that the paper-system can be tiring and tedious, and they tend to compromise the quality of their service because there is too much paperwork for them to do.
    \item Lack of funds

    The Department of Health, as the primary local agency in charge of health, releassd Formula One for Health in 2005, which aims to reform the local health sector. There is a notable focus on finances, as the DOH planned to make the spending of public funds more transparent in order to secure more support for the health sector. In the Philippines, the health sector can be divided into two: public and private. It is unique among Southeast Asian countries because most of its medical services come from the private sector. The percentage of money allocated by the government into healthcare is already low and steadily decreasing every year, so the public sector caters more to patients who lack the money to go to private hospitals \cite{nguyen2008analysis}.
    \item Lack of decentralization

    The health sector is also far from being decentralized, with great reliance on central rather than peripheral authorities. Data are gathered locally, then passed on step by step to organizations until they reach DOH; DOH then comes up with health policies based on these data, and these flow back down. This setup is inefficient and ineffective. The World Health Organizafion (WHO) suggested at the 1978 conference at Alma-Ata that district health systems would benefit from decentralization. Two attempts were made to reform the system in the Philippines, in 1991 and 1998, but these failed because they chose to keep recording data on paper.

    In any endeavor to promote medical informatics, the DOH is a major influencer from a larger perspective, as shown by centralization: it inevitably influences decision-making because it is in charge of the report submission process. Local government units also have to be taken into account. If their support is not obtained or if tension exists, then it is unlikely that a system would be successful \cite{premji2012implementing}.
    \item Network Infrastructure

    Technology has become increasingly cheap, with most developing countries having easy access to internet \cite{fraser2005implementing}. However, not everyone in the Philippines is comfortable with computers, much less even own them; phones are more prevalent, but these may not be able to download complex applications. Moreover, several remote areas, where health workers are needed, do not have electricity \cite{marcelo2007health}. If a system is to be built, then all the available resources must be accounted for, and cost must be weighed against benefits.
\end{enumerate}

\subsection{Case Study: CHITS}

A system called the Community Health Information Tracking System (CHITS) was implemented by the Medical Informatics Unit of UP Manila. This could store electronic health records as well as set clinic appointments, and this was tested out in at least 36 health centers. An observation of their process of introducing the system to its users yields several psychological insights \cite{premji2012implementing}

Healthcare providers as well as government officials were carefully guided through a step-by-step training; when exposed to the technology so that they became comfortable enough with it, they realized how much more efficient and easy it is to use. Those unfamiliar with computers warmed up to it by playing simple games then moving on to more complex tasks \cite{premji2012implementing}.

The learning curve remained steep, however. Every day for the next few months, healthcare providers used the paper-based system during busy hours at work, then practiced the electronic system when they had more time. Some of them still remained anxious, but most of them got past their discomfort and became proficient at it. This gave them a sense of pride about the new skill that they'd mastered. They also experienced immense relief at being able to print everything at the end of the day rather than filling up all the forms by hand. As a result, patients were handled with less delay. As one provider said: " In the manual, paper-based system it [took] 10 minutes to find a patient record but now it takes seconds to find." They also trusted the data more, so when they noticed patterns that hinted at an outbreak, they told the head physician right away \cite{premji2012implementing}.
