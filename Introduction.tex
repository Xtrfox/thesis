\chapter{Introduction}


Doctors, nurses, and other healthcare practitioners have to deal with huge amounts of information whenever they engage with their work, from pinpointing the various symptoms of a disease to assessing the condition of a patient. Traditionally, patient information is noted down by hand on paper and then collated into folders. This method can get extremely disorganized, with information piling up after years and years; it can also be inconvenient and susceptible to loss, as usually no other copy is made.

The discipline of medical informatics was developed due to the increasing need for a better handling of medical information, as access to healthcare broadens and healthcare institutions grow in size. Patients also have a greater demand for transparency and desire to play a more proactive role with regards to their health. Technology thus comes into play. The current thrust of medical informatics involves developing desktop or mobile applications that give access to databases of patient data. Physical records now become electronic medical records (EMR), so that data such as a patient’s weight or what his treatment was for which appointment can be accessed with a click or a swipe of the finger. This focuses on the patient-specific type of information in medical informatics. The other type is knowledge-based, which is science-oriented rather than people-oriented--for example, an AI that can specify the likely disease as long as one can input the symptoms displayed by the patient \cite{williamr.hersh2002}.

This research is centered on patient-specific information, as it aims to improve on ShineOS+, a mobile and desktop health informatics application that is meant to be used by nurses, doctors, administrators, and other healthcare practitioners as well as patients. Aside from saving patient data, it also has several other features such as keeping track of consultations and appointments, generating reports required by the Department of Health, and streamlining referrals so EMRs can be passed easily from one doctor to another without loss of information. This has immense benefits, some of which may not be as obvious: having the exact prescription noted down digitally can help patients monitor their intake better, and nurses can input their notes so patient profiling becomes more well-rounded. Its use can even be taken beyond the hospital to a disaster setting, where rapid medical help is most needed but where confusion and chaos pose huge obstacles.

One of the strengths of the application is its flexibility--not only is it meant to be used in every digital platform, from Mac OS to Windows to Android, but it also caters to a wide array of users and adapts accordingly. The emphasis is on improving the workflow of the health community, which is one of the major challenges in health informatics, as doctors and nurses can find it difficult and time-consuming to integrate using an app while dealing with patients. Aside from creating an organized system that can present information clearly and thoroughly, the other side of the equation must also be considered: how to design user experience so that users will feel comfortable using it? To a certain extent, this has already been addressed by ShineOS+, as it also allows synchronization between multiple devices and its code is open-source so as to allow innovation and constant responsiveness to the needs of its users. A core tenet of its philosophy is user-centric design, and this seems to have been implemented--a layperson can easily use the app without getting confused and the design is colorful and pleasing.

However, the mobile version--which is still in beta--presents some problems that weren’t present in desktop, which was the original platform. Mobile users are less inclined to fill up long forms. The sheer amount of information presented in one view of the desktop app can become overwhelming when transferred onto mobile. It is very likely, then, that users would not interact significantly with the mobile app because of inconvenience and frustration. This is a significant problem. Phones are much more accessible than laptops, and healthcare practitioners especially need to have the option of being able to enter and access data on the go. Additionally, part of the vision for the application involves turning over more control to the user--custom forms and themes, custom displays, and so on. These have not been implemented yet, and would go a long way towards engaging users.

A possible solution to this problem would be gamification, where game concepts such as reward systems and statistics tracking are applied to a non-game setting in order to make it more compelling to users. This has been successfully used in a wide variety of contexts. A very common example is how customers are given a card so they can rack up points whenever they make a purchase; once they amass enough points, they get a freebie. A website called FreeRice has been quite successful: users answer a trivia quiz, and for every correct answer, the website donates ten grains of rice, as provided by sponsors. Aside from the short-term satisfaction of getting the answer, users also gain a more profound sense of pleasure from being altruistic, and this entices them to return to the website again and again. The most apparent tool used in gamification is giving out rewards. However, relying on this too much can result in an unhealthy infinite loop where the user craves more and more rewards. A more long-term approach would be what is called meaningful gamification, which includes elements such as play (having fun and not being paralyzed by the fear of failure), reflection (connecting the current setting with previous experience or being led to gain insight), and engagement (facilitating an easier and more natural connection between users. Ideally, a sense of narrative is also employed so the user feels as if she is on a quest or a journey \cite{scottnicholson2014}. The challenge faced by this research is, in essence, applying gamification effectively to medical informatics, more specifically through ShineOS+ mobile.

The challenge faced by this research is, in essence, applying gamification effectively to medical informatics, more specifically through ShineOS+ mobile.

\section{Research Questions}

This research aims to answer the question ``Can social media posts from Twitter be one of the determinants of disease outbreaks within the Philippines?''

Other subquestions are as follows:
\begin{itemize}
\item What are related keywords and phrases frequently mentioned by Twitter users regarding the three diseases, especially when they are prevalent across an area?
\item What attributes of social media posts are accessible and significant to be able to produce a population surveillance report?
\item How do we interface tweets with syndromic surveillance data from existing systems of the Department of Health as well as from electronic medical records (EMR)?
\item How do we develop a validated spatio-temporal epidemiological model that shows and forecasts possible outbreak areas based on Twitter posts?
\end{itemize}
