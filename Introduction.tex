\chapter{Introduction}


Doctors, nurses, and other healthcare practitioners have to deal with huge amounts of information whenever they engage with their work, from pinpointing the various symptoms of a disease to assessing the condition of a patient. Traditionally, patient information is noted down by hand on paper and then collated into folders. This method can get extremely disorganized, with information piling up after years and years; it can also be inconvenient and susceptible to loss, as usually no other copy is made.

The discipline of medical informatics was developed due to the increasing need for a better handling of medical information, as access to healthcare broadens and healthcare institutions grow in size. Patients also have a greater demand for transparency and desire to play a more proactive role with regards to their health. Technology thus comes into play. The current thrust of medical informatics involves developing desktop or mobile applications that give access to databases of patient data. Physical records now become electronic medical records (EMR), so that data such as a patient’s weight or what his treatment was for which appointment can be accessed with a click or a swipe of the finger. This focuses on the patient-specific type of information in medical informatics. The other type is knowledge-based, which is science-oriented rather than people-oriented--for example, an AI that can specify the likely disease as long as one can input the symptoms displayed by the patient \cite{williamr.hersh2002}.

This research is centered on patient-specific information, as it aims to improve on ShineOS+, a mobile and desktop health informatics application that is meant to be used by nurses, doctors, administrators, and other healthcare practitioners as well as patients. Aside from saving patient data, it also has several other features such as keeping track of consultations and appointments, generating reports required by the Department of Health, and streamlining referrals so EMRs can be passed easily from one doctor to another without loss of information. This has immense benefits, some of which may not be as obvious: having the exact prescription noted down digitally can help patients monitor their intake better, and nurses can input their notes so patient profiling becomes more well-rounded. Its use can even be taken beyond the hospital to a disaster setting, where rapid medical help is most needed but where confusion and chaos pose huge obstacles.

One of the strengths of the application is its flexibility--not only is it meant to be used in every digital platform, from Mac OS to Windows to Android, but it also caters to a wide array of users and adapts accordingly. The emphasis is on improving the workflow of the health community, which is one of the major challenges in health informatics, as doctors and nurses can find it difficult and time-consuming to integrate using an app while dealing with patients. Aside from creating an organized system that can present information clearly and thoroughly, the other side of the equation must also be considered: how to design user experience so that users will feel comfortable using it? To a certain extent, this has already been addressed by ShineOS+, as it also allows synchronization between multiple devices and its code is open-source so as to allow innovation and constant responsiveness to the needs of its users. A core tenet of its philosophy is user-centric design, and this seems to have been implemented--a layperson can easily use the app without getting confused and the design is colorful and pleasing.

However, the mobile version--which is still in beta--presents some problems that weren’t present in desktop, which was the original platform. Mobile users are less inclined to fill up long forms. The sheer amount of information presented in one view of the desktop app can become overwhelming when transferred onto mobile. It is very likely, then, that users would not interact significantly with the mobile app because of inconvenience and frustration. This is a significant problem. Phones are much more accessible than laptops, and healthcare practitioners especially need to have the option of being able to enter and access data on the go. Additionally, part of the vision for the application involves turning over more control to the user--custom forms and themes, custom displays, and so on. These have not been implemented yet, and would go a long way towards engaging users.

A possible solution to this problem would be gamification, where game concepts such as reward systems and statistics tracking are applied to a non-game setting in order to make it more compelling to users. This has been successfully used in a wide variety of contexts. A very common example is how customers are given a card so they can rack up points whenever they make a purchase; once they amass enough points, they get a freebie. A website called FreeRice has been quite successful: users answer a trivia quiz, and for every correct answer, the website donates ten grains of rice, as provided by sponsors. Aside from the short-term satisfaction of getting the answer, users also gain a more profound sense of pleasure from being altruistic, and this entices them to return to the website again and again. The most apparent tool used in gamification is giving out rewards. However, relying on this too much can result in an unhealthy infinite loop where the user craves more and more rewards. A more long-term approach would be what is called meaningful gamification, which includes elements such as play (having fun and not being paralyzed by the fear of failure), reflection (connecting the current setting with previous experience or being led to gain insight), and engagement (facilitating an easier and more natural connection between users. Ideally, a sense of narrative is also employed so the user feels as if she is on a quest or a journey \cite{scottnicholson2014}. The challenge faced by this research is, in essence, applying gamification effectively to medical informatics, more specifically through ShineOS+ mobile.

The challenge faced by this research is, in essence, applying gamification effectively to medical informatics, more specifically through ShineOS+ mobile.

\section{Research Questions}

This research aims to answer the question ``How can gamification theory be applied to ShineOS+, an electronic medical record application, to incentivize healthcare practitioners to use the application regularly?''

Other subquestions are as follows:
\begin{itemize}
\item What specific gaming elements can it include to encourage users to use the application?
\item How significant is the change of user engagement and user satisfaction after gamification is applied to the application?

\end{itemize}

\section{Research Objectives}

This research aims to make ShineOS+, a medical informatics mobile app, more appealing and engaging to users through gamification, so that it can fulfill its vision of enhancing the workflow of healthcare practitioners. Examples of gamification might be incorporating reward systems, eliciting a greater sense of community, or giving the user mote autonomy by allowing him greater control over the display (e.g. customizable UI). These modifications will be applied to the app, the new version of which should ready for release by the end of this research.

\section{Scope and Limitations}

As ShineOS+ has a desktop and mobile version, the research would cover applying gamification of the application to both versions. The study will generally focus on enriching the overall user experience by introducing gaming mechanics to reward users for completing tasks listed in the app. In events the app has missing features that would potentially enrich the overall user experience, the research would also cover adding extra functionality to the existing application.
After the implementation of gamification, the study will also analyze if there was a significant improvement in user engagement and user satisfaction.

One of the limitations of this research is that it will not deal with the acquisition of non-virtual rewards, which essentially requires making a deal with pharmaceutical companies. Overall, the research would only limit itself in creating an overall better user experience of the existing application.

\section{Significance of the Study}

Should the gamification prove to be successful, ShineOS+ would extend its user reach and compel more healthcare practitioners to transfer their data onto electronic medical records, especially as this practice is uncommon in the Philippines. This would also make collaboration easier since patient information can be passed from one account to another. Aside from the obvious impact of ease of workflow and a more organized system, long-term use--which is to be encouraged by gamification--might also change the outlook of practitioners on data-keeping. By being able to input the data easily on their phone, they might get into the habit of being more meticulous with checking their patients and jotting down their observations. Transparency would also be encouraged. When the patient asks for an explanation, the practitioner will be able to show the patient’s personalized record from previous consultations, which would lead to them having a more open and informed exchange.

Because electronic medical records are not yet the norm and practitioners may feel reluctant to try them, preferring to stick with the familiar, gamification would provide a bridge. Applying gamification to the mobile version of ShineOS+ can easily be carried over to the desktop version--to the whole cross-platform system, in fact, even MyShine, the corresponding app for patients that is still under development. Instead of seeing record-keeping as a chore, users would come to find it fun and even interesting and would eventually engage in it for their own reasons.

The application of gamification to an electronic medical record application is also a novel idea. While gamification has been applied to various contexts, medical informatics technology tends to be austere, putting most of its focus on the functionality; combining gamification and medical informatics should thus yield interesting results, as they are not commonly put together, and would lead to a more profound understanding, if not new insights, about gamification itself. Rather than short-term change, what this research aims for is for users to consult ShineOS+ regularly--to have it become an integral part of their everyday workflow--which only high-level gamification would achieve.

