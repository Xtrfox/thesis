\section{Research Questions}

This research aims to answer the question ``How can gamification theory be applied to ShineOS+, an electronic medical record application, to incentivize healthcare practitioners to use the application regularly?''

Other subquestions are as follows:
\begin{itemize}
\item What specific gaming elements can it include to encourage users to use the application?
\item How significant is the change of user engagement and user satisfaction after gamification is applied to the application?

\end{itemize}

\section{Research Objectives}

This research aims to make ShineOS+, a medical informatics mobile app, more appealing and engaging to users through gamification, so that it can fulfill its vision of enhancing the workflow of healthcare practitioners. Examples of gamification might be incorporating reward systems, eliciting a greater sense of community, or giving the user mote autonomy by allowing him greater control over the display (e.g. customizable UI). These modifications will be applied to the app, the new version of which should ready for release by the end of this research.

\section{Scope and Limitations}

As ShineOS+ has a desktop and mobile version, the research would cover applying gamification of the application to both versions. The study will generally focus on enriching the overall user experience by introducing gaming mechanics to reward users for completing tasks listed in the app. In events the app has missing features that would potentially enrich the overall user experience, the research would also cover adding extra functionality to the existing application.
After the implementation of gamification, the study will also analyze if there was a significant improvement in user engagement and user satisfaction.

One of the limitations of this research is that it will not deal with the acquisition of non-virtual rewards, which essentially requires making a deal with pharmaceutical companies. Overall, the research would only limit itself in creating an overall better user experience of the existing application.

\section{Significance of the Study}

Should the gamification prove to be successful, ShineOS+ would extend its user reach and compel more healthcare practitioners to transfer their data onto electronic medical records, especially as this practice is uncommon in the Philippines. This would also make collaboration easier since patient information can be passed from one account to another. Aside from the obvious impact of ease of workflow and a more organized system, long-term use--which is to be encouraged by gamification--might also change the outlook of practitioners on data-keeping. By being able to input the data easily on their phone, they might get into the habit of being more meticulous with checking their patients and jotting down their observations. Transparency would also be encouraged. When the patient asks for an explanation, the practitioner will be able to show the patient’s personalized record from previous consultations, which would lead to them having a more open and informed exchange.

Because electronic medical records are not yet the norm and practitioners may feel reluctant to try them, preferring to stick with the familiar, gamification would provide a bridge. Applying gamification to the mobile version of ShineOS+ can easily be carried over to the desktop version--to the whole cross-platform system, in fact, even MyShine, the corresponding app for patients that is still under development. Instead of seeing record-keeping as a chore, users would come to find it fun and even interesting and would eventually engage in it for their own reasons.

The application of gamification to an electronic medical record application is also a novel idea. While gamification has been applied to various contexts, medical informatics technology tends to be austere, putting most of its focus on the functionality; combining gamification and medical informatics should thus yield interesting results, as they are not commonly put together, and would lead to a more profound understanding, if not new insights, about gamification itself. Rather than short-term change, what this research aims for is for users to consult ShineOS+ regularly--to have it become an integral part of their everyday workflow--which only high-level gamification would achieve.
